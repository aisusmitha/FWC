\documentclass[a4paper,10pt]{article}
\usepackage{graphicx}
\usepackage{amsmath}
\usepackage[a4paper,top=1in,bottom=1in,left=0.8in,right=0.8in]{geometry}
\usepackage{amsmath, amssymb}
\usepackage{xcolor}
\usepackage{fancyhdr}
\definecolor{cyanblue}{RGB}{0, 140, 200}
\setstretch{1.1}
\begin{document}

\pagestyle{empty} % Start with empty page style

\thispagestyle{fancy} % Apply fancy style only to the first page
\fancyhf{} % Clear header and footer
\renewcommand{\headrulewidth}{0pt} % Remove header rule

\fancyhead[L]{% Left header
        \includegraphics[width=8cm, height=1.7cm]{img3.png} 
        }
\fancyhead[R]{% Right header
    Name: K.Saisusmitha  \\
    Batch: COMETFWC018 \\
    Date: 16 May 2025 } 
    \vspace{1cm}
\begin{center}

    {\LARGE \textbf{\textcolor{cyanblue}{\\ PAIR OF LINEAR EQUATIONS\\ In TWO VARIABLES}}}
\end{center}
  \vspace{1cm}   
\textbf{\textcolor{cyanblue}{Example 6:}}In a shop the cost of 2 pencils and 3 erasers is ₹9 and the cost of 4 pencils and 6 erasers is ₹18. Find the cost of each pencil and each eraser.

\textbf{\textcolor{cyanblue}{Solution :}} The pair of linear equations formed were:
\begin{align}
2x + 3y &= 9 \tag{1}\\
4x + 6y &= 18 \tag{2}
\end{align}
We first express the value of \( x \) in terms of \( y \) from Equation (1), to get
\begin{align}
x = \frac{9 - 3y}{2} \tag{3}
\end{align}
Now we substitute this value of \( x \) in Equation (2), to get
\begin{align*}
4 \left( \frac{9 - 3y}{2} \right) + 6y &= 18 \\
18 - 6y + 6y &= 18 \\
\Rightarrow 18 &= 18
\end{align*}
This statement is true for all values of \( y \). However, we do not get a specific value of \( y \) as a solution. Therefore, we cannot obtain a specific value of \( x \). This situation has arisen because both the given equations are the same. 
Therefore, Equations (1) and (2) have infinitely many solutions. We cannot find a unique cost of a pencil and an eraser, because there are many common solutions to the given situation.

\noindent
\textbf{\textcolor{cyanblue}{Example 7:}}Two rails are represented by the equations:
[
x + 2y - 4 &= 0 \quad \text{and} \quad 2x + 4y - 12 &= 0 
]
will the rails cross each other?

\textbf{\textcolor{cyanblue}{Solution :}}The pair of linear equations formed were:
\begin{align}
x + 24 -4 &=0 \tag{1}\\
2x +4y -12 &=0 \tag{2}
\end{align}
 We express \( x \) in terms of \( y \) from Equation (1), to get
\[
x = 4 - 2y
\]
Now, we substitute this value of \( x \) in Equation (2), to get
\begin{align*}
2(4 - 2y) + 4y - 12 &= 0 \\
8 - 4y + 4y - 12 &= 0 \\
-4 &= 0 
\end{align*}
Which is a False statement.\\
Therefore the  equations donot have a common solution.so,the rails will not cross each other. 

\end{document}
