\documentclass[12pt]{article}
\usepackage[a4paper,top=1.5cm,bottom=2.5cm,left=3cm,right=2.5cm,headheight=15pt,headsep=10pt]{geometry}
\usepackage{enumitem}
\usepackage{amsmath}
\usepackage{graphicx}
\usepackage{xcolor}
\usepackage[table]{xcolor}
\usepackage{colortbl}
\usepackage{array}
\usepackage{titlesec}
\usepackage{fancyhdr}
\usepackage{helvet}
\usepackage{float}
\usepackage{multicol}
\usepackage{etoolbox}

\renewcommand{\familydefault}{\sfdefault}

\definecolor{cyanblue}{RGB}{0,180,200}

\pagestyle{empty} % Start with empty page style

\thispagestyle{fancy} % Apply fancy style only to the first page
\fancyhf{} % Clear header and footer
\renewcommand{\headrulewidth}{1pt} 
\makeatletter
\patchcmd{\headrule}
  {\hrule}
  {\color{cyanblue}\hrule}
  {}{}
\makeatother
\fancyhead[L]{% Left header
        \includegraphics[width=8cm, height=1.7cm]{img3.png} 
        }
\fancyhead[R]{% Right header
    Name: K.Saisusmitha \\
    Batch: COMETFWC018 \\
    Date: 16 may 2025
}
\fancyfoot[C]{{Reprint 2025-26}}
\begin{enumerate}
\begin{document}
\vspace{1cm}
\textcolor{cyanblue}{\textbf{Solution :}} Let us denote the number of pants by $x$ and the number of skirts by $y$. Then the equations formed are:
\[
\begin{aligned}
y &= 2x - 2 \quad \quad \quad \quad (1) \\
y &= 4x - 4 \quad \quad \quad \quad (2)
\end{aligned}
\]

\noindent
\hangindent=1.5em
Let us draw the graphs of\\ Equations (1) and (2) by finding two\\ solutions for each of the equations.\\
\noindent
\hangindent=1.5em
They are given in \textbf{Table 3.3}.

\vspace{0.4cm}
\begin{figure}[H]
\noindent
\begin{minipage}[t]{0.48\textwidth}
\begin{center}
\textcolor{cyanblue}{\textbf{Table 3.3}}
\end{center}

\renewcommand{\arraystretch}{1.5}

% First Table
\begin{tabular}{
    |>{\columncolor{cyanblue!30}}m{2.8cm}|
     >{\columncolor{cyanblue!30}\centering\arraybackslash}m{1.5cm}|
     >{\columncolor{cyanblue!30}\centering\arraybackslash}m{1.5cm}|
}
\hline
\textcolor{black}{\textbf{x}} & 2 & 0 \\
\hline
\textcolor{black}{\textbf{y = 2x - 2}} & 2 & -2 \\
\hline
\end{tabular}

\vspace{0.5cm}

% Second Table
\begin{tabular}{
    |>{\columncolor{cyanblue!30}}m{2.8cm}|
     >{\columncolor{cyanblue!30}\centering\arraybackslash}m{1.5cm}|
     >{\columncolor{cyanblue!30}\centering\arraybackslash}m{1.5cm}|
}
\hline
\textcolor{black}{\textbf{x}} & 0 & 1 \\
\hline
\textcolor{black}{\textbf{y = 4x - 4}} & -4 & 0 \\
\hline
\end{tabular}
\end{minipage}
\hfill
\begin{minipage}[t]{0.48\textwidth}
\vspace{0pt}
\begin{figure}
    \centering
    \includegraphics[width=5cm,height=6cm]{img4.jpg}
    \label{fig:enter-label}
\end{figure}
{\textcolor{cyanblue}{\textbf{Fig. 3.2}}}
\end{minipage}
\end{figure}

\vspace{0.2cm}
\noindent
Plot the points and draw the lines passing through them to represent the equations, as shown in Fig. 3.2.

\vspace{0.2cm}
\noindent
The two lines intersect at the point (1, 0). So, $x = 1$, $y = 0$ is the required solution of the pair of linear equations, i.e., the number of pants she purchased is 1 and she did not buy any skirt.

\vspace{0.2cm}
\noindent
\textbf{\textit{Verify}} the answer by checking whether it satisfies the conditions of the given problem.

% Reset header for next page (page 29 style)
% Exercise 3.1 begins
\begin{center}
\textcolor{cyanblue}{\textbf{\large EXERCISE 3.1}}
\end{center}
\end{enumerate}
\begin{enumerate}
\item Form the pair of linear equations in the following problems, and find their solutions graphically.

\begin{enumerate}
\item 10 students of Class X took part in a Mathematics quiz. If the number of girls is 4 more than the number of boys, find the number of boys and girls who took part in the quiz.
\end{enumerate}

\newpage
\noindent
\textcolor{cyanblue}{\textbf{PAIR OF LINEAR EQUATIONS IN TWO VARIABLES}}
\hfill
\textcolor{cyanblue}{\textbf{29}}\\[-0.5em]
\textcolor{cyanblue}{\rule{\textwidth}{0.8pt}}

\noindent
\begin{enumerate}[resume]
\item 5 pencils and 7 pens together cost ₹ 50, whereas 7 pencils and 5 pens together cost ₹ 46. Find the cost of one pencil and that of one pen.
\end{enumerate}

\item On comparing the ratios \( \dfrac{a_1}{a_2}, \dfrac{b_1}{b_2} \) and \( \dfrac{c_1}{c_2} \), find out whether the lines representing the following pairs of linear equations intersect at a point, are parallel or coincident:

\begin{multicols}{2}
\begin{enumerate}[label=(\roman*)]
    \item \(5x - 4y + 8 = 0\)\\
          \(7x + 6y - 9 = 0\)

    \item \(9x + 3y + 12 = 0\)\\
          \(18x + 6y + 24 = 0\)

    \item \(6x - 3y + 10 = 0\)\\
          \(2x - y + 9 = 0\)
\end{enumerate}
\end{multicols}

\item On comparing the ratios \( \dfrac{a_1}{a_2}, \dfrac{b_1}{b_2} \) and \( \dfrac{c_1}{c_2} \), find out whether the following pair of linear equations are consistent, or inconsistent.

\begin{multicols}{2}
\begin{enumerate}[label=(\roman*)]
    \item \(3x + 2y = 5\), \quad \(2x - 3y = 7\)

    \item \(2x - 3y = 8\), \quad \(4x - 6y = 9\)

    \item \( \dfrac{3}{2}x + \dfrac{5}{3}y = 7 \), \quad \(9x - 10y = 14\)

    \item \(5x - 3y = 11\), \quad \(10x + 6y = -22\)

    \item \( \dfrac{4}{3}x + 2y = 8 \), \quad \(2x + 3y = 12\)
\end{enumerate}
\end{multicols}

\item Which of the following pairs of linear equations are consistent/inconsistent? If consistent, obtain the solution graphically:

\begin{enumerate}[label=(\roman*)]
    \item \(x + y = 5,\quad 2x + 2y = 10\)
    \item \(x - y = 8,\quad 3x - 3y = 16\)
    \item \(2x + y - 6 = 0,\quad 4x - 2y - 4 = 0\)
    \item \(2x - 2y - 2 = 0,\quad 4x - 4y - 5 = 0\)
\end{enumerate}

\item Half the perimeter of a rectangular garden, whose length is 4 m more than its width, is 36 m. Find the dimensions of the garden.

\item Given the linear equation \(2x + 3y - 8 = 0\), write another linear equation in two variables such that the geometrical representation of the pair so formed is:
\begin{enumerate}[label=(\roman*)]
    \item intersecting lines
    \item parallel lines
    \item coincident lines
\end{enumerate}

\item Draw the graphs of the equations \(x - y + 1 = 0\) and \(3x + 2y - 12 = 0\). Determine the coordinates of the vertices of the triangle formed by these lines and the \textit{x}-axis, and shade the triangular region.
\end{enumerate}
\vfill
\begin{center}
{Reprint 2025–26}
\end{center}
\newpage
% Page 33 Content

\noindent\textcolor{cyanblue}{\textbf{PAIR OF LINEAR EQUATIONS IN TWO VARIABLES}}\hfill\textcolor{cyanblue}{\textbf{33}}\\[-0.5em]
\textcolor{cyanblue}{\rule{\textwidth}{0.8pt}}

\vspace{1em}

\noindent
i.e.,\quad $8 - 12 = 0$ \\
i.e.,\quad $-4 = 0$ \\
which is a false statement. \\
Therefore, the equations do not have a common solution. So, the two rails will not cross each other.

\vspace{1em}

\begin{center}
    \textcolor{cyanblue}{\textbf{EXERCISE 3.2}}
\end{center}

\begin{enumerate}
    \item Solve the following pair of linear equations by the substitution method:
    \begin{enumerate}
        \item $x + y = 14$, \quad $x - y = 4$
        \item $s - t = 3$, \quad $\dfrac{s}{3} + \dfrac{t}{2} = 6$
        \item $3x - y = 3$, \quad $9x - 3y = 9$
        \item $0.2x + 0.3y = 1.3$, \quad $0.4x + 0.5y = 2.3$
        \item $\sqrt{2}x + \sqrt{3}y = 0$, \quad $\sqrt{3}x - \sqrt{8}y = 0$
        \item $\dfrac{3x}{2} - \dfrac{5y}{3} = 2$, \quad $\dfrac{x}{3} + \dfrac{y}{2} = \dfrac{13}{6}$
    \end{enumerate}

    \item Solve $2x + 3y = 11$ and $2x - 4y = -24$ and hence find the value of ‘$m$’ for which $y = mx + 3$.

    \item Form the pair of linear equations for the following problems and find their solution by substitution method:
    \begin{enumerate}
        \item The difference between two numbers is 26 and one number is three times the other. Find them.
        \item The larger of two supplementary angles exceeds the smaller by 18 degrees. Find them.
        \item The coach of a cricket team buys 7 bats and 6 balls for ₹ 3800. Later, she buys 3 bats and 5 balls for ₹ 1750. Find the cost of each bat and each ball.
        \item The taxi charges in a city consist of a fixed charge together with the charge for the distance covered. For a distance of 10 km, the charge paid is ₹ 105 and for a journey of 15 km, the charge paid is ₹ 155. What are the fixed charges and the charge per km? How much does a person have to pay for travelling a distance of 25 km?
        \item A fraction becomes $\dfrac{9}{11}$, if 2 is added to both the numerator and the denominator. If 3 is added to both the numerator and the denominator it becomes $\dfrac{5}{6}$. Find the fraction.
    \end{enumerate}
\end{enumerate}

\vfill
\begin{center}
Reprint 2025–26
\end{center}

\end{document}
