\documentclass[a4paper,12pt]{article}
\usepackage{tikz}
\usetikzlibrary{circuits.logic.US, positioning}
\usepackage[utf8]{inputenc}
\usepackage{amsmath}
\usepackage{geometry}
\geometry{margin=1in}
\usepackage{caption}
\usepackage{subcaption}
\usepackage{xcolor}
\usepackage{fancyhdr}
\usepackage{array}
\usepackage{float}
\definecolor{darkskyblue}{rgb}{0.0, 0.5, 1.0}
\definecolor{skyblue}{RGB}{135, 206, 235}

\geometry{a4paper, top=0.7in, left=1in, right=1in, bottom=1in}

\begin{document}
\pagestyle{empty} % Start with empty page style

\thispagestyle{fancy} % Apply fancy style only to the first page
\fancyhf{} % Clear header and footer
\renewcommand{\headrulewidth}{0pt} % Remove header rule

\fancyhead[L]{% Left header
        \includegraphics[width=8cm, height=1.7cm]{img3.png} 
                }

\fancyhead[R]{% Right header
    Name: K.Saisusmitha \\
    Batch: COMETFWC018 \\
    Date: 16 may 2025
}
\vspace{1cm}
\begin{center}

    {\LARGE \textbf{\textcolor{darkskyblue}{\\  GATE QUESTION \\ IN 2010 Q31}}}
\end{center}

\vspace{-1cm} %adjust vertical space
\section*{\textcolor{blue}{\\Question}}

\noindent\textbf{Q.42} The logic gate circuit shown in the adjoining figure realizes the function
\vspace{0.5cm}

\begin{center}
    \includegraphics[width=0.65\textwidth]{i.jpg}
\end{center}

\vspace{0.3cm}

\noindent\textbf{Options:}
\noindent
(A) XOR \hfill (B) XNOR \hfill (C) Half adder \hfill (D) Full adder
\section*{Solution:}

\vspace{0.5em}

The given circuit has two inputs: \( X \) and \( Y \).

\vspace{0.5em}

The circuit uses two \textbf{AND gates} and one \textbf{OR gate}.

\vspace{0.5em}

From the diagram, the inputs are connected to the AND gates in a crossed fashion:

\begin{enumerate}
  \item The first AND gate receives inputs \( X \) and \( \overline{Y} \) → Output: \( X \cdot \overline{Y} \)
  \item The second AND gate receives inputs \( \overline{X} \) and \( Y \) → Output: \( \overline{X} \cdot Y \)
\end{enumerate}

The outputs of both AND gates are then connected to an OR gate, which gives:
\[
Z = (X \cdot \overline{Y}) + (\overline{X} \cdot Y)
\]

This is the standard Boolean expression for the \textbf{XOR} operation.

\vspace{1em}

\noindent Therefore, the circuit realizes the function:
\[
Z = X \oplus Y
\]

\vspace{0.3cm}
\noindent\textbf{Answer:} (A) XOR

\end{document}

