\documentclass[a4paper,12pt]{article}
\usepackage{tikz}
\usetikzlibrary{circuits.logic.US, positioning}
\usepackage[utf8]{inputenc}
\usepackage{amsmath}
\usepackage{geometry}
\geometry{margin=1in}
\usepackage{caption}
\usepackage{subcaption}
\usepackage{xcolor}
\usepackage{fancyhdr}
\usepackage{array}
\usepackage{float}
\definecolor{darkskyblue}{rgb}{0.0, 0.5, 1.0}
\definecolor{skyblue}{RGB}{135, 206, 235}

\geometry{a4paper, top=0.7in, left=1in, right=1in, bottom=1in}

\begin{document}
\pagestyle{empty} % Start with empty page style

\thispagestyle{fancy} % Apply fancy style only to the first page
\fancyhf{} % Clear header and footer
\renewcommand{\headrulewidth}{0pt} % Remove header rule

\fancyhead[L]{% Left header
        \includegraphics[width=8cm, height=1.7cm]{img3.png} 
        }

\fancyhead[R]{% Right header
    Name: K.Saisusmitha \\
    Batch: COMETFWC018 \\
    Date: 16 may 2025
}
\vspace{1cm}
\begin{center}

    {\LARGE \textbf{\textcolor{darkskyblue}{\\  GATE QUESTION \\ CSE 2010 Q31}}}
\end{center}

\vspace{-1cm} %adjust vertical space
\section*{\textcolor{blue}{\\Question}}

\noindent
\textbf{Q.31} \textit{What is the boolean expression for the output $f$ of the combinational logic circuit of NOR gates given below?}

\vspace{1cm}

\begin{center}
\begin{tikzpicture}[circuit logic US, scale=1, every node/.style={transform shape}]

% Top NOR gate (P, Q)
\node (nor1) [nor gate, draw, logic gate inputs=nn, anchor=input 1] at (0,0) {};
\node[left=0.5cm of nor1.input 1] (P) {$P$};
\node[left=0.5cm of nor1.input 2] (Q) {$Q$};
\draw (P) -- (nor1.input 1);
\draw (Q) -- (nor1.input 2);

% Second NOR gate (Q, R)
\node (nor2) [nor gate, draw, logic gate inputs=nn, anchor=input 1] at (0,-2) {};
\node[left=0.5cm of nor2.input 1] (Q) {$Q$};
\node[left=0.5cm of nor2.input 2] (R1) {$R$};
\draw (Q) -- (nor2.input 1);
\draw (R1) -- (nor2.input 2);

% Third NOR gate (P, R)
\node (nor3) [nor gate, draw, logic gate inputs=nn, anchor=input 1] at (0,-4) {};
\node[left=0.5cm of nor3.input 1] (P2) {$P$};
\node[left=0.5cm of nor3.input 2] (R2) {$R$};
\draw (P2) -- (nor3.input 1);
\draw (R2) -- (nor3.input 2);

% Fourth NOR gate (Q, R) - again
\node (nor4) [nor gate, draw, logic gate inputs=nn, anchor=input 1] at (0,-6) {};
\node[left=0.5cm of nor4.input 1] (Q2) {$Q$};
\node[left=0.5cm of nor4.input 2] (R3) {$R$};
\draw (Q2) -- (nor4.input 1);
\draw (R3) -- (nor4.input 2);

% Fifth NOR gate (joins nor1 and nor2)
\node (nor5) [nor gate, draw, logic gate inputs=nn, anchor=input 1] at (3,-1) {};
\draw (nor1.output) -- ++(0.5,0) |- (nor5.input 1);
\draw (nor2.output) -- ++(0.5,0) |- (nor5.input 2);

% Sixth NOR gate (joins nor3 and nor4)
\node (nor6) [nor gate, draw, logic gate inputs=nn, anchor=input 1] at (3,-5) {};
\draw (nor3.output) -- ++(0.5,0) |- (nor6.input 1);
\draw (nor4.output) -- ++(0.5,0) |- (nor6.input 2);

% Final NOR gate (joins nor5 and nor6)
\node (nor7) [nor gate, draw, logic gate inputs=nn, anchor=input 1] at (6,-3) {};
\draw (nor5.output) -- ++(0.5,0) |- (nor7.input 1);
\draw (nor6.output) -- ++(0.5,0) |- (nor7.input 2);

% Output f
\draw (nor7.output) -- ++(1,0) node[right] {$f$};

\end{tikzpicture}
\end{center}

\vspace{1cm}

\noindent
(A) $\overline{Q + R}$ \hfill (B) $\overline{P + \overline{Q}}$ \hfill (C) $\overline{P + R}$ \hfill (D) $\overline{P + Q + R}$
\section*{Solution:}

We are given a logic circuit made entirely of NOR gates, and we are to determine the Boolean expression for the output \( f \).

\vspace{0.5cm}
\textbf{Recall:} A NOR gate performs the operation:
\[
A \downarrow B = \overline{A + B}
\]

\subsection*{Step-by-step Analysis}

\begin{enumerate}
    \item Let the intermediate outputs of the NOR gates be labeled as follows:
    \begin{align*}
    A &= \overline{P + Q} \\
    B &= \overline{Q + R} \\
    C &= \overline{P + R} \\
    D &= \overline{Q + R}
    \end{align*}

    \item The next layer of NOR gates takes inputs as follows:
    \begin{align*}
    E &= \overline{A + B} = \overline{\overline{P + Q} + \overline{Q + R}} \\
    F &= \overline{C + D} = \overline{\overline{P + R} + \overline{Q + R}}
    \end{align*}

    \item The final output is:
    \[
    f = \overline{E + F} = \overline{\,\overline{\overline{P + Q} + \overline{Q + R}} + \overline{\overline{P + R} + \overline{Q + R}}\,}
    \]

    \item On simplifying or testing the expression through truth tables, we find that the output is high only when both \( Q = 0 \) and \( R = 0 \). This corresponds to:
    \[
    f = \overline{Q + R}
    \]
\end{enumerate}

\subsection*{Final Answer:}
\[
\boxed{f = \overline{Q + R}} \quad \text{(Option A)}
\]

\end{document}

