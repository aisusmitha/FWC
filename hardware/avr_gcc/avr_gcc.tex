\documentclass[a4paper,12pt]{article}
\usepackage{tikz}
\usetikzlibrary{circuits.logic.US, positioning}
\usepackage[utf8]{inputenc}
\usepackage{amsmath}
\usepackage{geometry}
\geometry{margin=1in}
\usepackage{caption}
\usepackage{subcaption}
\usepackage{xcolor}
\usepackage{fancyhdr}
\usepackage{array}
\usepackage{float}
\definecolor{darkskyblue}{rgb}{0.0, 0.5, 1.0}
\definecolor{skyblue}{RGB}{135, 206, 235}

\geometry{a4paper, top=0.7in, left=1in, right=1in, bottom=1in}

\begin{document}
\pagestyle{empty} % Start with empty page style

\thispagestyle{fancy} % Apply fancy style only to the first page
\fancyhf{} % Clear header and footer
\renewcommand{\headrulewidth}{0pt} % Remove header rule

\fancyhead[L]{% Left header
        \includegraphics[width=8cm, height=1.7cm]{img3.png} 
                }

\fancyhead[R]{% Right header
    Name: K.Saisusmitha \\
    Batch: COMETFWC018 \\
    Date: 16 may 2025
}
\vspace{1cm}
\begin{center}

    {\LARGE \textbf{\textcolor{darkskyblue}{\\  GATE QUESTION \\ PH 2010 Q31}}}
\end{center}

\vspace{-1cm} %adjust vertical space
\section*{\textcolor{blue}{\\Question}}

\noindent\textbf{Q.41} For any set of inputs, A and B, the following circuits give the same output, Q, except one.\\
\textbf{Which one is it?}

\vspace{1em}

\begin{center}
    \includegraphics[width=0.9\textwidth]{i2.jpg} % Change filename if needed

\end{center}

\vspace{1em}

\noindent\textbf{Answer: (C)}

\vspace{1em}

\noindent\textbf{Explanation:}

\begin{enumerate}
    \item Option (A): Implements the XNOR logic using NOR and NOT gates. Correct.
    \item Option (B): Uses De Morgan's law. Equivalent to XNOR. Correct.
    \item Option (C): The circuit implements XOR due to how the gates are configured. This is different from the others.\\
    Hence, this circuit gives a different output.
    \item Option (D): This also simplifies to XNOR. Correct.
\end{enumerate}

\noindent Therefore, the circuit in option \textbf{(C)} does not match the output of others.

\end{document}

